\subsection{Subproblemas}

\subsubsection{Semántica del OPT}
Dado el punto $(f, c)$ y un $k$, nuestro $OPT(f, c, k)$ devolverá la cantidad máxima de celdas ocupadas por las que puede pasar en un recorrido de ida desde $(f, c)$
hasta $(n, n)$ y uno de vuelta desde $(n, n)$ hasta $(f - k, c + k)$. A la ida, solo se podrá mover hacia la derecha y hacia abajo, mientras que a la vuelta lo hará
hacia la izquierda y hacia arriba, moviéndose una celda a la vez.
De no existir un recorrido válido, devolverá $-\infty$.

\subsubsection{Movimientos y diagonales}
Dado el problema B)I, nuestro algoritmo calcula el $OPT$ de ida desde la celda de salida $(f, c)$ hasta $(n, n)$ y el de vuelta desde $(n, n)$ hasta $(f - k, c + k)$ simultáneamente,
encarando la vuelta como si el recorrido $(n, n) \to (f - k, c + k)$ fuese de ida, es decir, desde $(f - k, c + k)$ hasta $(n, n)$.

Por definición del problema, en los viajes de ida los movimientos permitidos son hacia la derecha y hacia abajo, moviéndose una casilla por vez. Dado que hay dos viajes simultáneos,
las instancias de movimiento posibles son permutaciones de derecha y abajo, resultando en 4 posibilidades:
\begin{enumerate}
	\item $i_1$: derecha e $i_2$: derecha
	\item $i_1$: derecha e $i_2$: abajo
	\item $i_1$: abajo e $i_2$: derecha
	\item $i_1$: abajo e $i_2$: abajo
\end{enumerate}

Dado que los movimientos son de una casilla a la vez para ambos recorridos y estos se mueven a la vez, para todo $j$ los recorridos siempre están en una misma diagonal,
donde la suma del índice $f$ y el índice $c$ es una constante. En todo movimiento, cada recorrido aumentará el índice $f$ o $c$ en uno. Sea:
\[
	f + c = (f - k) + (c + k)
\]
Posición inicial:
\[
	f + c + j = (f - k) + (c + k) + j
\]
con $j$ siendo la cantidad de movimientos. Dada esta relación, podemos determinar que ambos recorridos se encuentran en $(n, n)$ en la misma cantidad de pasos.

\subsubsection{Celdas compartidas y prohibidas}
Dado que ambos recorridos siempre están en la misma diagonal, puede darse el caso de que tengan celdas en común en el recorrido que maximiza el paso por celdas ocupadas.
De ser así, el recorrido total solo contará la primera vez que se pase por una celda ocupada, pudiendo resultar en un recorrido erróneo.
Para controlar estos casos, nuestro $OPT$ contará con una función auxiliar llamada \texttt{celda}, la cual asigna valores a las celdas, siendo libres, ocupadas o prohibidas,
asignándoles $0$, $1$ o $-\infty$, respectivamente.

Sabiendo que ambos recorridos son simultáneos y con la misma cantidad de movimientos, si los recorridos tienen celdas comunes, ambos estarán en ellas a la vez,
facilitando la comparación y la asignación correcta de valores para cada celda. En cuanto a las celdas prohibidas, el valor asignado es $-\infty$
de forma que al llamar a la función $\max(,)$, nunca se elegirá un camino que pase por una celda prohibida.

\subsubsection{Maximización simultánea}
Por lo desarrollado en la Sección 2, ya sabemos que el $OPT$, en caso de pasar por las mismas celdas ocupadas, las considerará solo una vez en $i_1$.
Siendo así, cuando se llame a la función $\max$ para $i_2$, priorizará el camino que maximice las celdas ocupadas tomando las celdas comunes como si estuvieran libres.

Por el funcionamiento de $\max(,)$ y la asignación de valores de \texttt{Ocupadas}, nuestro $OPT$ tendrá un orden de prioridad tal que:
\[
	\text{prohibida} < \text{libre} < \text{ocupada}
\]
donde se desprende que $-\infty < 0 < 1$, maximizando así el paso por celdas ocupadas para ambos recorridos.
