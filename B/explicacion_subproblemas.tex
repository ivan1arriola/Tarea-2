\subsubsection{Identificación de subproblemas}

\paragraph{Semántica del Problema}
En este problema, consideramos un recorrido doble: uno de ida desde una celda 
de inicio $(f, c)$ hasta la celda $(n, n)$, y otro de vuelta desde $(n, n)$ 
hasta una celda de destino $(f - k, c + k)$. El objetivo de $OPT(f, c, k)$ 
es calcular la máxima cantidad de celdas ocupadas por las que se puede pasar 
en ambos recorridos. Si no es posible realizar un recorrido válido, 
el valor devuelto será $-\infty$.

\paragraph{Subproblema 1: Movimientos Simultáneos en el Recorrido de Ida y Vuelta}

En este problema, nuestro algoritmo calcula el valor $OPT$ de ida desde la celda de salida 
$(f, c)$ hasta $(n, n)$, y el de vuelta desde $(n, n)$ hasta $(f - k, c + k)$ simultáneamente. 
Para ello, consideramos el recorrido de vuelta como si fuese de ida, es decir, desde 
$(f - k, c + k)$ hasta $(n, n)$.

Por definición, en el recorrido de ida, los movimientos permitidos son hacia la derecha y 
hacia abajo, moviéndose una casilla por vez. Dado que hay dos recorridos que se realizan 
simultáneamente, las combinaciones posibles de movimiento se pueden describir como 
permutaciones de derecha y abajo, resultando en cuatro opciones:

\begin{enumerate}
    \item $i_1$: derecha y $i_2$: derecha
    \item $i_1$: derecha y $i_2$: abajo
    \item $i_1$: abajo y $i_2$: derecha
    \item $i_1$: abajo y $i_2$: abajo
\end{enumerate}

Dado que los movimientos son de una casilla a la vez para ambos recorridos y se realizan 
simultáneamente, en todo momento, para un paso $j$, los recorridos permanecen en la misma 
diagonal. Esto se debe a que la suma de los índices $f$ y $c$ es constante. 

En cada movimiento, cada recorrido aumentará el índice $f$ o $c$ en uno. Definimos:

\begin{align*}
\text{Posición inicial:} & \quad f + c = (f - k) + (c + k) \\
\text{Nueva posición:} & \quad f + c + j = (f - k) + (c + k) + j
\end{align*}

donde $j$ es la cantidad de movimientos realizados. A partir de esta relación, podemos 
concluir que ambos recorridos se encuentran en la posición $(n, n)$ tras realizar la misma 
cantidad de pasos y debemos considerar cómo esta restricción afecta 
los movimientos disponibles y la optimización de celdas ocupadas.

\paragraph{Subproblema 2: $k = 0$ y Celdas Prohibidas}
Un segundo subproblema se presenta cuando ambos recorridos pasan por las mismas celdas ocupadas, es decir, 
cuando $k = 0$. En estos casos, es fundamental asegurar que las celdas compartidas se cuenten solo una vez.

Para manejar esta situación, se implementa una función auxiliar llamada \texttt{celda}. 
Esta función toma como argumentos la fila, la columna y el valor de $k$, y 
devuelve la cantidad de celdas ocupadas en el conjunto 
\[
\{(f, c), (f - k, c + k)\}.
\]

Además, la función toma en cuenta si las celdas de ida y de vuelta son la misma.

Por último, la función $OPT(f, c, k)$ debe evitar cualquier recorrido que pase por celdas prohibidas, 
evitando estas celdas en la respuesta final.

\paragraph{Subproblema 3: Maximización simultánea de ambos recorridos}

El algoritmo $OPT$ considera que, si ambos recorridos pasan por las mismas celdas ocupadas, 
estas solo se contabilizan una vez en el recorrido de ida ($i_1$). 

Por lo tanto, al calcular el máximo para el recorrido de vuelta ($i_2$), 
el algoritmo prioriza el camino que maximiza el número de celdas ocupadas, 
tratando las celdas comunes como si fueran celdas libres.

En función de cómo opera la función $\max(,)$ y de la asignación de valores a las celdas ocupadas, 
el algoritmo $OPT$ seguirá un orden de prioridad definido, donde:
\[
\text{prohibida} < \text{libre} < \text{ocupada}
\]
Esto implica que $-\infty < 0 < 1 < 2$, permitiendo así maximizar el número de celdas ocupadas alcanzadas en 
cada diagonal del recorrido.
