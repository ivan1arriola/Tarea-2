\subsubsection{Correctitud de la Recurrencia}

Para demostrar la correctitud de la recurrencia, primero vamos a demostrar que la recurrencia termina y luego vamos a demostrar que la solución es correcta.

\paragraph{Terminación de la Recurrencia}

Para demostrar que la recurrencia termina, vamos a mostrar que eventualmente llegamos a un caso base.

Para cualquier celda $(i, j)$ y $k$ cualquiera, la recurrencia puede caer en uno de los siguientes casos:
\begin{itemize}
    \item $i = n$ y $j = n$ [1],
    \item $i \in \{0, n+1\}$ [2],
    \item $j \in \{0, n+1\}$ [3],
    \item $k$ es tal que $k < 0$ o $k > \min\{n - j, i - 1\}$ [4], es decir, $k$ es tal que se sale de los límites de la cuadrícula.
    \item Ninguno de los anteriores [5].
\end{itemize}

En el caso [1], la recurrencia llega a un caso base, ya que se ha llegado a la celda $(n, n)$ y el algoritmo termina.

En los casos [2], [3] y [4], la recurrencia llega a un caso base, ya que se ha llegado a una celda fuera de la cuadrícula y se llama a la función \texttt{celda} sin más iteraciones.

En el caso [5], se llama a la función \texttt{celda}, que devuelve un valor que se suma a la llamada recursiva de la recurrencia. En cada llamada recursiva, se va a mover a la derecha o hacia abajo, creciendo monotonamente la fila o la columna. Como solo existen $n$ filas y $n$ columnas, la cantidad de llamadas recursivas no puede ser mayor a $2n$ porque eventualmente se llegará a uno de los casos [1], [2], [3] o [4].

\paragraph{Correctitud de la Recurrencia}
Vamos a demostrar que la recurrencia es correcta por inducción en las filas, columnas y $k$.

\paragraph{Caso Base} 
Para el caso base, vamos a demostrar que el valor de $OPT(f,c,k)$ es correcto para los siguientes casos:
\begin{itemize}
    \item $OPT(n,n,k)$ con $k$ cualquiera,
    \item $OPT(i, j', k)$, con $j' \in \{0, n+1\}$, $i$ cualquiera, $k$ cualquiera,
    \item $OPT(i', j, k)$, con $i' \in \{0, n+1\}$, $j$ cualquiera, $k$ cualquiera,
    \item $OPT(i, j, k)$, con $i$ y $j$ cualquiera y $k$ tal que $k < 0$ o $k > \min\{n - j, i - 1\}$.
\end{itemize}

Demostración: 
\begin{itemize}
    \item Para $OPT(n, n, k)$, la recurrencia devuelve $celda(n,n, k)$ que es $1$ si la celda $(n, n)$ está ocupada y $k = 0$, $0$ si la celda $(n, n)$ está libre y $k = 0$ y $-\infty$ en cualquier otro caso. Semanticamente es correcto porque con $k$ distinto de $0$ se sale de los límites de la cuadrícula. Con $k = 0$ se llega a la celda $(n, n)$. Y el valor de la celda es correcto. Nunca se alcanza el caso de $celda(n, n, k) = 2$ porque no puede sumar $2$ celdas ocupadas ya que solo puede visitar $(n, n)$.

    \item Para $OPT(i, j', k)$ con $j' \in \{0, n+1\}$, la recurrencia devuelve $celda(i, j', k)$. En este caso, la recurrencia termina y devuelve el valor $-\infty$ porque se salió de los límites de la cuadrícula.

    \item Para $OPT(i', j, k)$ con $i' \in \{0, n+1\}$, la recurrencia devuelve $celda(i', j, k)$. En este caso, la recurrencia termina y devuelve el valor $-\infty$ porque se salió de los límites de la cuadrícula.

    \item Para $OPT(i, j, k)$ con $i$ y $j$ cualquiera y $k$ tal que $k < 0$ o $k > \min\{n - j, i - 1\}$, la recurrencia devuelve $celda(i, j, k)$. En este caso, la recurrencia termina y devuelve el valor $-\infty$ porque se salió de los límites de la cuadrícula.
\end{itemize}

Con esto, demostramos que la recurrencia es correcta para los casos base.

\paragraph{Caso Inductivo}: 

\textbf{Hipótesis Inductiva:} Supongamos que la recurrencia es correcta para $OPT(i+1, j, k)$, $OPT(i, j+1, k)$, $OPT(i, j, k+1)$ y $OPT(i, j, k-1)$.

\textbf{Paso Inductivo:} Demostraremos que la recurrencia es correcta para $OPT(i, j, k)$.

Por definición de $OPT(i, j, k)$, tenemos que:
\[
OPT(i, j, k) = celda(i, j, k) + \max\left\{OPT(i+1, j, k), OPT(i, j+1, k), OPT(i, j, k+1), OPT(i, j, k-1)\right\}
\]

Por hipótesis inductiva, sabemos que los valores de $OPT(i+1, j, k)$, $OPT(i, j+1, k)$, $OPT(i, j, k+1)$ y $OPT(i, j, k-1)$ son correctos.

\begin{itemize}
    \item Si $celda(i, j, k) = 0$, entonces $OPT(i, j, k) = \max\left\{
        \begin{array}{l}
            OPT(i+1, j, k), \\
            OPT(i, j+1, k), \\
            OPT(i, j, k+1), \\
            OPT(i, j, k-1)
        \end{array}
    \right\}$. Lo cual está bien porque no suma ninguna celda ocupada.
    
    \item Si $celda(i, j, k) = 1$, entonces $OPT(i, j, k) = 1 + \max\left\{
        \begin{array}{l}
            OPT(i+1, j, k), \\
            OPT(i, j+1, k), \\
            OPT(i, j, k+1), \\
            OPT(i, j, k-1)
        \end{array}
    \right\}$. Lo cual está bien porque suma una celda ocupada.
    
    \item Si $celda(i, j, k) = 2$, entonces $OPT(i, j, k) = 2 + \max\left\{
        \begin{array}{l}
            OPT(i+1, j, k), \\
            OPT(i, j+1, k), \\
            OPT(i, j, k+1), \\
            OPT(i, j, k-1)
        \end{array}
    \right\}$. Lo cual está bien porque suma dos celdas ocupadas.
    
    \item Si $celda(i, j, k) = -\infty$, entonces $OPT(i, j, k) = -\infty + \max\left\{
        \begin{array}{l}
            OPT(i+1, j, k), \\
            OPT(i, j+1, k), \\
            OPT(i, j, k+1), \\
            OPT(i, j, k-1)
        \end{array}
    \right\} = -\infty$. Lo cual está bien porque no es una celda válida para realizar el recorrido.
\end{itemize}


Con esto, demostramos que la recurrencia es correcta para los casos inductivos.

Por lo tanto, la recurrencia es correcta.



