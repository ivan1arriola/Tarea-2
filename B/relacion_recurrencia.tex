\subsection{Relación de recurrencia}

Primero definimos la función \texttt{celda}:
\[
	\text{celda}(f, c, k) =
	\begin{cases}
		-\infty & \text{si } k < 0 \text{ o } k > \min\{n - c, f - 1\},                            \\
		        & \text{o si } (f, c) \text{ o } (f-k, c+k) \text{ es una celda prohibida},        \\ \\
		0       & \text{si } (f, c) \text{ y } (f-k, c+k) \text{ son celdas libres},               \\  \\
		1       & \text{si } (f, c) \text{ es una celda ocupada y } (f-k, c+k) \text{ es libre},   \\
		        & \text{o si } (f-k, c+k) \text{ es una celda ocupada y } (f, c) \text{ es libre}, \\
		        & \text{o si } (f, c) \text{ es ocupada y } k = 0.                                 \\ \\
		2       & \text{si } (f, c) \text{ y } (f-k, c+k) \text{ son celdas ocupadas y } k \neq 0,
	\end{cases}
\]


Luego, definimos la relación de recurrencia para $OPT(f, c, k)$:
Dado un punto $(f, c)$ y un $k$, nuestro $OPT(f, c, k)$ devolverá la cantidad máxima de celdas ocupadas por las que puede pasar en un recorrido de ida desde $(f, c)$
hasta $(n, n)$ y uno de vuelta desde $(n, n)$ hasta $(f - k, c + k)$. A la ida, solo se podrá mover hacia la derecha y hacia abajo, mientras que a la vuelta lo hará
hacia la izquierda y hacia arriba, moviéndose una celda a la vez.

\[
	OPT(f, c, k) =
	\begin{cases}
		\text{Celda}(f, c, k) & \text{si } f = n \text{ y } c = n, \\
		                      & \text{o si } f \in \{0, n+1\},     \\
		                      & \text{o si } c \in \{0, n+1\},     \\ \\
		\text{Celda}(f, c, k) + \max\left\{
		\begin{array}{l}
			OPT(f, c+1, k),   \\
			OPT(f+1, c, k),   \\
			OPT(f+1, c, k-1), \\
			OPT(f, c+1, k-1)
		\end{array}
		\right\}              & \text{en otro caso}.
	\end{cases}
\]