\documentclass[a4paper,12pt]{article}
\usepackage[utf8]{inputenc}
\usepackage[spanish]{babel}
\usepackage{amsmath}
\usepackage{amsthm}
\usepackage{geometry}          % Para controlar márgenes y tamaños
\usepackage{algpseudocode}
\usepackage{algorithm}
\usepackage{hyperref}          % Paquete para metadata y enlaces interactivos

% Configuración de márgenes equivalente
\geometry{
    left=0.5in,
    right=0.5in,
    top=0.5in,
    bottom=1in
}

\setlength{\parindent}{0em}    % No sangría en los párrafos
\setlength{\parskip}{2ex}      % Espacio entre párrafos

\renewcommand\thesection{\Alph{section}} % Secciones con letras
\renewcommand\thesubsection{\Roman{subsection}} % Subsecciones con números romanos
\renewcommand\thesubsubsection{\arabic{subsubsection}} % Subsubsecciones con números normales


% Metadata del PDF
\hypersetup{
    pdfauthor={Grupo 149},      % Nombre del autor
    pdftitle={Tarea 2 de Programación 3},      % Título del documento
    pdfsubject={Recorridos en cuadrículas},    % Asunto
    pdfkeywords={recorridos, algoritmos, programación}, % Palabras clave
    colorlinks=true,                           % Activar enlaces interactivos
    linkcolor=blue                             % Color de los enlaces
}

\title{Tarea 2 de Programación 3}
\author{Grupo 149}  
\date{Lunes, 21 de octubre de 2024}

\begin{document}

\maketitle

% Tabla con Nombre, Apellido y Cedula de los 4 integrantes

\newpage

\textbf{Problema: Recorrido de Ida y Vuelta}
\section{Ejercicio A}

\subsection{ Relación de recurrencia para $OPT(f, c)$ y correctitud.}
\subsubsection{Definición de $OPT$}

La función $OPT$ es la función que, dada una fila $f$ y una columna $c$, devuelve la máxima cantidad de celdas ocupadas por las que se puede pasar en un recorrido desde \( (f, c) \) hasta \( (n, n) \).

Se define como:

\[
	OPT(f, c) =
	\begin{cases}
		celda(f, c)                                             & \text{si } f = n \text{ y } c = n, \\
		                                                        & \text{o } f = n+1,                 \\
		                                                        & \text{o } c = n+1,                 \\
		                                                        & \text{o } f = 0,                   \\
		                                                        & \text{o } c = 0,                   \\
		celda(f, c) + \max\left(OPT(f, c+1), OPT(f+1, c)\right) & \text{en otro caso}.
	\end{cases}
\]

Donde:

\[
	celda(f,c) =
	\begin{cases}
		1       & \text{si } (f,c) \text{ es una celda ocupada,}   \\
		0       & \text{si } (f,c) \text{ es una celda libre,}     \\
		-\infty & \text{si } (f,c) \text{ es una celda prohibida.}
	\end{cases}
\]


\subsubsection{Justificación de la Correctitud}

La relación de recurrencia termina al alcanzar la celda $(n,n)$ o al llegar a la fila $n+1$ o a la columna $n+1$. 
Al empezar en $(f,c)$, siempre nos moveremos hacia abajo o hacia la derecha. No es posible movernos infinitamente hacia abajo, ya que la fila $n+1$ tiene todas las celdas prohibidas, haciendo que $OPT(n+1,c)$ siempre sea $-\infty$ sin más llamadas recursivas. Lo mismo ocurre al movernos infinitamente a la derecha, ya que se alcanza la columna $n+1$, que también tiene todas las celdas prohibidas. 
Si comenzamos en la fila o columna $0$, la ejecución termina en ese momento. 

\subsubsection{Correctitud de los valores de $OPT$}

Sea $T(n)$ un tablero de $0$ a $n+1$ filas y de $0$ a $n+1$ columnas, donde las filas y columnas $0$ y $n+1$ contienen valores prohibidos.

Vamos a demostrar que los valores calculados por $OPT$ son correctos mediante inducción.

\paragraph{Paso Base:} 
\underline{Caso 1: $(f,c) = (n,n)$} 

Por definición de $OPT$:
\[
OPT(n,n) = celda(n,n) 
\]
Tenemos tres casos posibles:
\begin{itemize}
    \item Si $(n,n)$ es libre, el valor será $0$, lo cual es correcto porque no existen celdas ocupadas por alcanzar.
    \item Si $(n,n)$ está ocupada, el valor será $1$, lo cual es correcto porque alcanzamos solamente la celda $(n,n)$ y estaba ocupada.
    \item Si $(n,n)$ es prohibida, el valor será $-\infty$, lo cual es correcto porque no es posible alcanzar alguna celda.
\end{itemize}

\underline{Caso 2: Alcanzamos una fila o columna $0$ o $n+1$}

En estos casos, por definición de $OPT$ y de $T(n)$, los valores serán $-\infty$. Sin importar cómo se haya llegado a esas celdas, representan correctamente el hecho de que no son celdas válidas para realizar el recorrido.

\paragraph{Paso Inductivo}: \\
\textbf{Hipótesis Inductiva:} Supongamos que los valores de $OPT(f+1,c)$ y $OPT(f,c+1)$ son correctos.\\
\textbf{Tesis:} Se puede calcular el valor de $OPT(f,c)$ correctamente.

\textbf{Demostración:} 

Si $f = n$ y $c = n$, o $f = n+1$, o $c = n+1$, o $f = 0$, o $c = 0$, estamos en el caso base.

En caso contrario, es decir, si $0 < f \leq n$ y $0 < c \leq n$, entonces, por la hipótesis inductiva, los valores de $OPT(f+1, c)$ y $OPT(f, c+1)$ han sido calculados correctamente en pasos anteriores. Resta probar que el valor de $OPT(f, c)$ es correcto.

\[
OPT(f, c) = celda(f, c) + \max\left(OPT(f, c+1), OPT(f+1, c)\right) 
\]

En cada $(f,c)$ que elija, debo decidir si moverme hacia abajo $(f+1)$ o hacia la derecha $(c+1)$. En el momento de elegir, puedo optar por el que alcance la mayor cantidad de celdas ocupadas, lo cual está representado por el $\max$. Estos valores son correctos por hipótesis. Para calcular la cantidad total de celdas ocupadas alcanzadas, solo falta sumar el valor de la propia celda $(f,c)$, representada con $celda(f,c)$.

\begin{itemize}
    \item Si es una celda prohibida, se sumará $-\infty$, lo que resultará siempre en $-\infty$ sin importar hacia qué lado me mueva, representando correctamente la imposibilidad de alcanzar celdas ocupadas desde esa posición.
    \item Si la celda está libre, no se suma más celdas ocupadas a $OPT$, lo cual es correcto.
    \item Si la celda está ocupada, se suma $1$ al valor de $OPT$, representando una celda más que es posible alcanzar.
\end{itemize}

Por lo tanto, el resultado de $OPT(f,c)$ es correcto en todos los casos.

\newpage


\subsection{Algoritmo iterativo para calcular $OPT(1, 1)$ utilizando recurrencia y su complejidad}

\subsubsection{Algoritmos $CalcularOPT$ y $celda$}
\begin{algorithm}
	\caption{CalcularOPT}
	\label{alg:calcularOPT}
	\textbf{Entrada:} Tablero $T[fila, col]$ donde cada celda $(fila, col)$ tiene el valor \{libre, ocupado, prohibido\}; $n$: tamaño del tablero \\
	\textbf{Salida:} $OPT[fila, col]$: cantidad máxima de celdas ocupadas que se pueden alcanzar
	\begin{algorithmic}[1]

		\State Crear matriz $OPT$

		\For{$k \gets 0$ \textbf{hasta} $n+1$}
		\State $OPT[k, n+1] \gets -\infty$
		\State $OPT[n+1, k] \gets -\infty$
		\State $OPT[k, 0] \gets -\infty$
		\State $OPT[0, k] \gets -\infty$
		\EndFor

		\For{$f \gets n$ \textbf{hasta} $1$} \Comment{Iterar sobre filas}
		\For{$c \gets n$ \textbf{hasta} $1$} \Comment{Iterar sobre columnas}
		\If{$(f, c) = (n, n)$} \Comment{Saltear la celda $(n, n)$}
		\State $OPT[n, n] \gets \text{celda}(n, n)$
		\EndIf
		\State $OPT[f, c] \gets \text{celda}(f, c) + \max(OPT[f+1, c], OPT[f, c+1])$
		\EndFor
		\EndFor

		\State  \Return $OPT$
	\end{algorithmic}
\end{algorithm}

Se usa el algoritmo auxiliar \texttt{celda}:

\begin{algorithm}
	\caption{celda}
	\label{alg:celda}
	\textbf{Entrada:} $(f, c)$: posición de la celda; $T[fila, col]$: tablero donde cada celda tiene un valor \{libre, ocupado, prohibido\} \\
	\textbf{Salida:} Valor de la celda en función de si está ocupada, libre o prohibida
	\begin{algorithmic}[1]
		\If{$T[f, c] = \text{ocupada}$}
		\State \Return $1$ \Comment{La celda está ocupada}
		\ElsIf{$T[f, c] = \text{libre}$}
		\State \Return $0$ \Comment{La celda está libre}
		\ElsIf{$T[f, c] = \text{prohibida}$}
		\State \Return $-\infty$ \Comment{La celda es prohibida}
		\EndIf
	\end{algorithmic}
\end{algorithm}


\subsubsection{Orden de Ejecución}

El algoritmo \texttt{celda} tiene un tiempo de ejecución de $O(1)$, ya que simplemente consulta una entrada específica en la matriz $T$ y realiza, a lo sumo, tres comparaciones. Por lo tanto, su ejecución no depende del tamaño de la entrada.

El algoritmo \texttt{CalcularOPT} se analiza a continuación:

\begin{itemize}
	\item La creación de la matriz $OPT$ es de orden $O(n^2)$,
	      ya que se requiere espacio para almacenar los valores en una matriz de tamaño $n \times n$.
	\item Las líneas 3, 4, 5 y 6 realizan asignaciones a valores de la matriz $OPT$,
	      cada una de las cuales es de orden $O(1)$.
	\item El bucle \texttt{for} que itera de 0 a $n+1$ es de orden $O(n)$, ya que efectivamente itera $n+1$ veces.
	\item La comparación en la línea 10 es de orden $O(1)$.
	\item La línea 13 consiste en una asignación de un valor calculado por \texttt{celda},
	      que tiene un tiempo de ejecución de $O(1)$. Además, el cálculo del máximo entre dos números
	      también se realiza en $O(1)$. Por lo tanto, esta línea es de orden $O(1)$ en total.
	\item Las líneas 8 y 9 contienen un bucle anidado que realiza un total de $n^2$ ejecuciones,
	      lo que da como resultado un tiempo de ejecución de $O(n^2)$.
	\item Retornar la matriz en la línea 16 es de orden $O(1)$.
\end{itemize}

En resumen, el algoritmo \texttt{CalcularOPT} tiene un tiempo de ejecución que se puede describir
como $O(n^2)$ debido a las operaciones más costosas dentro de los bucles anidados.

\newpage


\subsection{Algoritmo para hallar el recorrido óptimo y su complejidad}

\subsubsection{Algoritmo $recorrido$}

\begin{algorithm}
	\caption{recorrido}
	\label{alg:recorrido}
	\textbf{Entrada:} Tablero $T$, $n$: tamaño del tablero \\
	\textbf{Salida:} Camino de $(1,1)$ a $(n,n)$ que maximiza la cantidad de celdas ocupadas visitadas
	\begin{algorithmic}[1]
		\State Crear matriz $OPT \gets \text{CalcularOPT}(T, n)$
		\State Crear $Camino$, un conjunto ordenado
		\State Sea $(fila, col) \gets (1, 1)$

		\While{$(fila, col) \neq (n, n)$}
		\If{$OPT[fila, col] = -\infty$}
		\State \Return conjunto vacío \Comment{Termina ejecución aquí}
		\EndIf
		\State Agregar $(fila, col)$ a $Camino$

		\If{$OPT[fila+1, col] > OPT[fila, col+1]$}
		\State $(fila, col) \gets (fila+1, col)$
		\Else
		\State $(fila, col) \gets (fila, col+1)$
		\EndIf
		\EndWhile

		\State Agregar $(n, n)$ a $Camino$
		\State \Return $Camino$
	\end{algorithmic}
\end{algorithm}
\subsubsection{Orden de Ejecución}

El algoritmo \texttt{Recorrido} se analiza a continuación:
\begin{itemize}
	\item 1. La llamada a \texttt{CalcularOPT} tiene un tiempo de ejecución de $O(n^2)$.
	\item 2. La creación de Camino se puede hacer con una lista, que se puede crear en $O(1)$.
	\item 3. La asignación de $(fila, col)$ es de orden $O(1)$.
	\item 4. El bucle \texttt{while} se ejecuta hasta que $(fila, col) = (n, n)$, lo que puede ocurrir en el peor caso después de $2n$ iteraciones.
	      Por lo tanto, el bucle tiene un tiempo de ejecución de $O(n)$.
	\item 5. La comparación en la línea 5 es de orden $O(1)$.
	\item 6. Crear y retornar una lista vacía es de orden $O(1)$.
	\item 8. Agregar $(fila, col)$ a $Camino$ es de orden $O(1)$ porque agregar un elemento a una lista es una operación de tiempo constante.
	\item 9. La comparación en la línea 9 es de orden $O(1)$.
	\item 10. 12. La asignación de $(fila, col)$ es de orden $O(1)$.
	\item 15. Agregar $(n, n)$ a $Camino$ es de orden $O(1)$.
	\item 16. Retornar $Camino$ es de orden $O(1)$.
\end{itemize}

En resumen, el algoritmo \texttt{Recorrido} tiene un tiempo de ejecución que se puede describir
como $O(n^2)$ debido a las operaciones más costosas dentro de los bucles anidados.




\newpage
\section{Ejercicio B}

\subsection{Identificación de subproblemas, relacion de recurrencia y correctitud}
\subsubsection{Identificación de subproblemas}

\paragraph{Semántica del Problema}
En este problema, consideramos un recorrido doble: uno de ida desde una celda 
de inicio $(f, c)$ hasta la celda $(n, n)$, y otro de vuelta desde $(n, n)$ 
hasta una celda de destino $(f - k, c + k)$. El objetivo de $OPT(f, c, k)$ 
es calcular la máxima cantidad de celdas ocupadas por las que se puede pasar 
en ambos recorridos. Si no es posible realizar un recorrido válido, 
el valor devuelto será $-\infty$.

\paragraph{Subproblema 1: Movimientos Simultáneos en el Recorrido de Ida y Vuelta}

En este problema, nuestro algoritmo calcula el valor $OPT$ de ida desde la celda de salida 
$(f, c)$ hasta $(n, n)$, y el de vuelta desde $(n, n)$ hasta $(f - k, c + k)$ simultáneamente. 
Para ello, consideramos el recorrido de vuelta como si fuese de ida, es decir, desde 
$(f - k, c + k)$ hasta $(n, n)$.

Por definición, en el recorrido de ida, los movimientos permitidos son hacia la derecha y 
hacia abajo, moviéndose una casilla por vez. Dado que hay dos recorridos que se realizan 
simultáneamente, las combinaciones posibles de movimiento se pueden describir como 
permutaciones de derecha y abajo, resultando en cuatro opciones:

\begin{enumerate}
    \item $i_1$: derecha y $i_2$: derecha
    \item $i_1$: derecha y $i_2$: abajo
    \item $i_1$: abajo y $i_2$: derecha
    \item $i_1$: abajo y $i_2$: abajo
\end{enumerate}

Dado que los movimientos son de una casilla a la vez para ambos recorridos y se realizan 
simultáneamente, en todo momento, para un paso $j$, los recorridos permanecen en la misma 
diagonal. Esto se debe a que la suma de los índices $f$ y $c$ es constante. 

En cada movimiento, cada recorrido aumentará el índice $f$ o $c$ en uno. Definimos:

\begin{align*}
\text{Posición inicial:} & \quad f + c = (f - k) + (c + k) \\
\text{Nueva posición:} & \quad f + c + j = (f - k) + (c + k) + j
\end{align*}

donde $j$ es la cantidad de movimientos realizados. A partir de esta relación, podemos 
concluir que ambos recorridos se encuentran en la posición $(n, n)$ tras realizar la misma 
cantidad de pasos y debemos considerar cómo esta restricción afecta 
los movimientos disponibles y la optimización de celdas ocupadas.

\paragraph{Subproblema 2: $k = 0$ y Celdas Prohibidas}
Un segundo subproblema se presenta cuando ambos recorridos pasan por las mismas celdas ocupadas, es decir, 
cuando $k = 0$. En estos casos, es fundamental asegurar que las celdas compartidas se cuenten solo una vez.

Para manejar esta situación, se implementa una función auxiliar llamada \texttt{celda}. 
Esta función toma como argumentos la fila, la columna y el valor de $k$, y 
devuelve la cantidad de celdas ocupadas en el conjunto 
\[
\{(f, c), (f - k, c + k)\}.
\]

Además, la función toma en cuenta si las celdas de ida y de vuelta son la misma.

Por último, la función $OPT(f, c, k)$ debe evitar cualquier recorrido que pase por celdas prohibidas, 
evitando estas celdas en la respuesta final.

\paragraph{Subproblema 3: Maximización simultánea de ambos recorridos}

El algoritmo $OPT$ considera que, si ambos recorridos pasan por las mismas celdas ocupadas, 
estas solo se contabilizan una vez en el recorrido de ida ($i_1$). 

Por lo tanto, al calcular el máximo para el recorrido de vuelta ($i_2$), 
el algoritmo prioriza el camino que maximiza el número de celdas ocupadas, 
tratando las celdas comunes como si fueran celdas libres.

En función de cómo opera la función $\max(,)$ y de la asignación de valores a las celdas ocupadas, 
el algoritmo $OPT$ seguirá un orden de prioridad definido, donde:
\[
\text{prohibida} < \text{libre} < \text{ocupada}
\]
Esto implica que $-\infty < 0 < 1 < 2$, permitiendo así maximizar el número de celdas ocupadas alcanzadas en 
cada diagonal del recorrido.

\subsubsection{Relación de recurrencia}

Primero definimos la función \texttt{celda}:
\[
	\text{celda}(f, c, k) =
	\begin{cases}
		-\infty & \text{si } k < 0 \text{ o } k > \min\{n - c, f - 1\},                            \\
		        & \text{o si } (f, c) \text{ o } (f-k, c+k) \text{ es una celda prohibida},        \\ \\
		0       & \text{si } (f, c) \text{ y } (f-k, c+k) \text{ son celdas libres},               \\  \\
		1       & \text{si } (f, c) \text{ es una celda ocupada y } (f-k, c+k) \text{ es libre},   \\
		        & \text{o si } (f-k, c+k) \text{ es una celda ocupada y } (f, c) \text{ es libre}, \\
		        & \text{o si } (f, c) \text{ es ocupada y } k = 0.                                 \\ \\
		2       & \text{si } (f, c) \text{ y } (f-k, c+k) \text{ son celdas ocupadas y } k \neq 0,
	\end{cases}
\]


Luego, definimos la relación de recurrencia para $OPT(f, c, k)$:
Dado un punto $(f, c)$ y un $k$, nuestro $OPT(f, c, k)$ devolverá la cantidad máxima de celdas ocupadas por las que puede pasar en un recorrido de ida desde $(f, c)$
hasta $(n, n)$ y uno de vuelta desde $(n, n)$ hasta $(f - k, c + k)$. A la ida, solo se podrá mover hacia la derecha y hacia abajo, mientras que a la vuelta lo hará
hacia la izquierda y hacia arriba, moviéndose una celda a la vez.

\[
OPT(f, c, k) =
\begin{cases}
\text{Celda}(f, c, k) & \text{si } f = n \text{ y } c = n, \\ 
                      & \text{o si } f \in \{0, n+1\}, \\ 
                      & \text{o si } c \in \{0, n+1\}, \\ 
                      & \text{o si } k < 0 \text{ o } k > \min\{n - c, f - 1\}, \\ \\ 
\text{Celda}(f, c, k) + \max\left\{ 
\begin{array}{l}
OPT(f, c+1, k), \\ 
OPT(f+1, c, k), \\ 
OPT(f+1, c, k+1), \\ 
OPT(f, c+1, k-1) 
\end{array} 
\right\} & \text{en otro caso}.
\end{cases}
\]
\subsubsection{Correctitud de la Recurrencia}

Para demostrar la correctitud de la recurrencia, primero vamos a demostrar que la recurrencia
termina y luego vamos a demostrar que la solución es correcta.

\paragraph{Terminación de la Recurrencia}

Para demostrar que la recurrencia termina, vamos a demostrar que eventualmente llegamos a un caso base.

Para cualquier celda $(i, j)$ y $k$ cualquiera, la recurrencia puede caer en uno de los siguientes casos:
[1] $i = n$ y $j = n$,
[2] $i \in \{0, n+1\}$,
[3] $j \in \{0, n+1\}$,
[4] Ninguna de las anteriores.

En el caso [1], la recurrencia llega a un caso base, ya que se llegó a la celda $(n, n)$ y el algoritmo termina.

En el caso [2] y [3], la recurrencia llega a un caso base, ya que se llegó a una celda fuera de la cuadrícula y 
se llama a la función \texttt{celda} sin mas iteraciones.

En el caso [4], se llama a la función \texttt{celda} que devuelve un valor que se suma a la llamada recursiva
de la recurrencia. En cada llamada recursiva, se va a mover a la derecha o hacia abajo, creciendo monotonamente
la fila o la columna. Como solo existen $n$ filas y $n$ columnas, la cantidad de llamadas recursivas no puede ser
mayor a $2n$ porque eventualmente se llegará a uno de los casos [1], [2] o [3].

\paragraph{Correctitud de la Recurrencia}
Vamos a demostrar que la recurrencia es correcta por inducción en las filas, columnas y $k$.

\paragraph{Caso Base} 
Para el caso base, vamos a demostrar que la recurrencia es correcta para las celdas $(n, n)$, $(0, j)$, $(i, 0)$ y $(n+1, j)$, $(i, n+1)$.
con $i$ y $j$ entre $0$ y $n+1$ y $k$ entre $0$ y $n+1$.
Para la celda $(n, n)$, la recurrencia devuelve el valor de la celda, que es la cantidad de celdas ocupadas en la celda $(n, n)$.


\subsection{Algoritmo y Complejidad}
\subsubsection{Algoritmos}
\begin{algorithm}
    \caption{Inicializar OPT}
    \label{alg:inicializarOPT}
    \textbf{Entrada:} Cuadrícula $matriz[f, c]$ donde cada celda $(f, c)$ tiene el valor \{'O': ocupada, 'L': libre, 'X': prohibida\}, $n$: tamaño de la cuadrícula \\
    \textbf{Salida:} Tabla tridimensional $OPT[f, c, k]$ inicializada con los valores base

    \begin{algorithmic}[1]
        \State \textbf{Definir} $opt[N][N][N]$ \Comment{Tabla tridimensional para almacenar el valor de OPT}
        \State \textbf{Definir} $matriz[N][N]$ \Comment{Cuadrícula de entrada}

        \For{$f \gets 1$ \textbf{hasta} $n$} \Comment{Iterar sobre las filas}
            \For{$c \gets 1$ \textbf{hasta} $n$} \Comment{Iterar sobre las columnas}
                \For{$k \gets 0$ \textbf{hasta} $n$} \Comment{Iterar sobre $k$}
                    \State $opt[f][c][k] \gets -\infty$ \Comment{Inicializamos con $-\infty$}
                \EndFor
            \EndFor
        \EndFor
        
        \If{$matriz[n][n] = 'O'$} 
            \State $opt[n][n][0] \gets 1$ \Comment{Celda ocupada}
        \ElsIf{$matriz[n][n] = 'L'$}
            \State $opt[n][n][0] \gets 0$ \Comment{Celda libre}
        \Else
            \State $opt[n][n][0] \gets -\infty$ \Comment{Celda prohibida}
        \EndIf
    \end{algorithmic}
\end{algorithm}


\begin{algorithm}
    \caption{Llenar OPT}
    \label{alg:llenarOPT}
    \textbf{Entrada:} Cuadrícula $matriz[f, c]$ donde cada celda $(f, c)$ tiene el valor \{'O': ocupada, 'L': libre, 'X': prohibida\}, $n$: tamaño de la cuadrícula \\
    \textbf{Salida:} Tabla tridimensional $OPT[f, c, k]$ llenada con los valores calculados

    \begin{algorithmic}[1]
        \For{$f \gets n$ \textbf{hasta} $1$} \Comment{Iterar desde $(n, n)$ hasta $(1, 1)$}
            \For{$c \gets n$ \textbf{hasta} $1$} 
                \For{$k \gets 0$ \textbf{hasta} $n$}
                    \State $f_k \gets f - k$
                    \State $c_k \gets c + k$
                    
                    \If{$f_k < 1$ \textbf{or} $c_k > n$} \Comment{Ignorar si sale de la cuadrícula}
                        \State \textbf{continue}
                    \EndIf

                    \If{$matriz[f][c] = 'X'$ \textbf{or} $matriz[f_k][c_k] = 'X'$} \Comment{Verificar si alguna celda es prohibida}
                        \State $opt[f][c][k] \gets -\infty$
                        \State \textbf{continue}
                    \EndIf

                    \State $ocupadas \gets 0$
                    \If{$f = f_k$ \textbf{and} $c = c_k$} \Comment{Misma celda en ida y vuelta}
                        \State $ocupadas \gets (matriz[f][c] = 'O')? 1 : 0$
                    \Else
                        \If{$matriz[f][c] = 'O'$} 
                            \State $ocupadas \gets ocupadas + 1$
                        \EndIf
                        \If{$matriz[f_k][c_k] = 'O'$} 
                            \State $ocupadas \gets ocupadas + 1$
                        \EndIf
                    \EndIf
                    
                    \State $max\_valor \gets -\infty$ \Comment{Inicializamos el valor máximo como $-\infty$}
                    
                    \If{$c + 1 \leq n$ \textbf{and} $c_k + 1 \leq n$} \Comment{Movimiento 1: ambos caminos hacia la derecha}
                        \State $max\_valor \gets \max(max\_valor, opt[f][c + 1][k] + ocupadas)$
                    \EndIf

                    \If{$c + 1 \leq n$ \textbf{and} $f_k + 1 \leq n$} \Comment{Movimiento 2: ida derecha, vuelta abajo}
                        \State $max\_valor \gets \max(max\_valor, opt[f][c + 1][k - 1] + ocupadas)$
                    \EndIf

                    \If{$f + 1 \leq n$ \textbf{and} $c_k + 1 \leq n$} \Comment{Movimiento 3: ida abajo, vuelta derecha}
                        \State $max\_valor \gets \max(max\_valor, opt[f + 1][c][k + 1] + ocupadas)$
                    \EndIf

                    \If{$f + 1 \leq n$ \textbf{and} $f_k + 1 \leq n$} \Comment{Movimiento 4: ambos caminos hacia abajo}
                        \State $max\_valor \gets \max(max\_valor, opt[f + 1][c][k] + ocupadas)$
                    \EndIf

                    \State $opt[f][c][k] \gets max\_valor$ \Comment{Asignar el valor máximo a $opt[f][c][k]$}
                \EndFor
            \EndFor
        \EndFor
    \end{algorithmic}
\end{algorithm}

\begin{algorithm}
    \caption{Calcular Resultado}
    \label{alg:calcularResultado}
    \textbf{Entrada:} Cuadrícula $matriz[f, c]$ de tamaño $n$ \\
    \textbf{Salida:} $OPT[1, 1, 0]$: el resultado final calculado por la tabla OPT

    \begin{algorithmic}[1]
        \State Inicializar la tabla OPT con valores base usando \textbf{Inicializar OPT}
        \State Llenar la tabla OPT usando \textbf{Llenar OPT}
        \State \Return $opt[1][1][0]$
    \end{algorithmic}
\end{algorithm}

\newpage

% Tiempo de ejecución
\subsection{Análisis de Complejidad}
Para analizar el tiempo de ejecución de la función principal: \texttt{calcular\_resultado}, vamos a analizar por separado el tiempo de ejecución de las funciones que la componen:

\texttt{inicializar\_OPT(n)}: Esta función inicializa las tablas que luego almacenarán el valor óptimo de celdas ocupadas visitadas para cada valor de \( k \) (siendo \( k \) la posición en la que se encuentra la diagonal de la matriz). Al mismo tiempo, nos indica el valor óptimo para el caso base, tomando como referencia el tipo de celda de la matriz (libre, ocupada o prohibida).

Esta inicialización está compuesta por tres \texttt{FOR} anidados, dos de ellos recorren las filas y columnas de la matriz y el tercero nos asegura que este paso lo realizaremos para todos los posibles \( k \) que nos determinen la posición de una diagonal.

Los valores posibles de filas y columnas se encuentran en el rango de \(\{1,\ldots,n\}\) y los de \( k \) en \(\{1,\ldots,\min\{n-c,f-1\}\}\) (con \( k \) teniendo a lo sumo \( n-1 \) valores).

Por la regla del producto, los tres \texttt{FOR} anidados nos determinan:
\[
	O(n) \cdot O(n) \cdot O(n-1) = O(n^3)
\]

Las comparaciones usadas por el caso base son \( O(1) \). Por lo tanto, por la regla de la suma:
\[
	\text{Inicializar\_OPT es } O(1) + O(n^3) = O(n^3)
\]

\texttt{llenar\_OPT(n)}: Esta función tiene como fin el llenado de la tabla que contiene los valores óptimos de celdas ocupadas de dos recorridos distintos que parten desde la misma diagonal.

Al igual que en la primera función, tenemos tres \texttt{FOR} anidados que determinan un tiempo de ejecución de partida de \( O(n^3) \) (esta vez \( k \) toma valores desde \(\{1,\ldots,n\}\), pero el tiempo de ejecución final se mantiene).

Para determinar el tiempo de ejecución total, nos centraremos en el tiempo de ejecución del bloque que contiene el \texttt{FOR} más interno. En el cuerpo de este \texttt{FOR} solo se realizan comparaciones (basadas en los movimientos simultáneos posibles de los dos recorridos).

El tiempo de cada comparación (instrucción \texttt{IF}) es el tiempo de evaluar la condición, más el mayor de los tiempos de las instrucciones en el \texttt{THEN} y el \texttt{ELSE}. Sin embargo, cada instrucción \texttt{IF} que se encuentra realiza asignaciones o comparaciones de \( O(1) \).

Por la regla de la suma, todas estas comparaciones son \( O(1) \). Por lo tanto, el orden de \texttt{llenar\_OPT(n)} es el orden de los \texttt{FOR}:
\[
	\text{Llenar\_OPT es } O(n^3)
\]

Por último, la función \texttt{calcular\_resultado(n)} realiza la inicialización y el llenado de tablas. Su tiempo de ejecución se compone de:
\[
	\text{Calcular\_resultado es } O(n^3) + O(n^3) = O(n^3)
\]




\end{document}
