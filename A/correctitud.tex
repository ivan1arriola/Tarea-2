\subsubsection{Justificación de la Correctitud}

La relación de recurrencia termina al alcanzar la celda $(n,n)$ o al llegar a la fila $n+1$ o a la columna $n+1$. \\
Al empezar en \( (f,c) \), siempre nos moveremos hacia abajo o hacia la derecha. No es posible movernos infinitamente hacia abajo, ya que la fila $n+1$ tiene todas las celdas prohibidas, haciendo que \( OPT(n+1,c) \) siempre sea $-\infty$ sin más llamadas recursivas. Lo mismo ocurre al movernos infinitamente a la derecha, ya que se alcanza la columna $n+1$, que también tiene todas las celdas prohibidas.\\
Si comenzamos en la fila o columna $0$, la ejecución termina en ese momento.\\

\subsubsection{Correctitud de los valores de $OPT$}

Sea $T(n)$ un tablero de $0$ a $n+1$ filas y de $0$ a $n+1$ columnas,
donde las filas y columnas $0$ y $n+1$ contienen valores prohibidos. \\

Vamos a demostrar que los valores calculados por $OPT$ son correctos mediante inducción.\\

\textbf{Paso Base:} \\

\underline{Caso 1: $(f,c) = (n,n)$} \\\\
Por definición de $OPT$:
\[
	OPT(n,n) = celda(n,n)
\]
Tenemos tres casos posibles:
\begin{itemize}
	\item Si $(n,n)$ es libre, el valor será $0$, lo cual es correcto porque no existen celdas ocupadas por alcanzar.
	\item Si $(n,n)$ está ocupada, el valor será \( 1 \), lo cual es correcto porque alcanzamos solamente la celda $(n,n)$ y estaba ocupada.
	\item Si $(n,n)$ es prohibida, el valor será $-\infty$, lo cual es correcto porque no es posible alcanzar alguna celda.
\end{itemize}
\underline{Caso 2: Alcanzamos una fila o columna $0$ o $n+1$} \\

En estos casos, por definición de $OPT$ y de $T(n)$, los valores serán $-\infty$.
Sin importar cómo se haya llegado a esas celdas,
representan correctamente el hecho de que no son celdas válidas para realizar el recorrido.\\

\textbf{Paso Inductivo:}  \\
\underline{Hipótesis}: Hemos calculado correctamente los valores de \( OPT(f',c') \)
donde \( f' \geq f+1 \) y \( c' \geq c+1 \).\\

\underline{Tesis}: Podemos calcular correctamente el valor de \( OPT(f,c) \).\\

\quad

Demostración:

Si \( f = n \) y \( c = n \), o \( f = n+1 \), o \( c = n+1 \), o \( f = 0 \), o \( c = 0 \), estamos en el caso base.

En caso contrario, es decir, si \( 0 < f \leq n \) y \( 0 < c \leq n \),
entonces, por la hipótesis inductiva, los valores de \( OPT(f+1, c) \) y \( OPT(f, c+1) \) han sido calculados correctamente
en pasos anteriores. Resta probar que el valor de \( OPT(f, c) \) es correcto.

\[
	OPT(f, c) = celda(f, c) + \max\left(OPT(f, c+1), OPT(f+1, c)\right)
\]

En cada $(f,c)$ que elija, debo decidir si moverme hacia abajo $(f+1)$ o hacia la derecha $(c+1)$.
En el momento de elegir, puedo optar por el que alcance la mayor cantidad de celdas ocupadas,
lo cual está representado por el $max$. Estos valores son correctos por hipótesis.
Para calcular la cantidad total de celdas ocupadas alcanzadas,
solo falta sumar el valor de la propia celda $(f,c)$, representada con $celda(f,c)$.

\begin{itemize}
	\item Si es una celda prohibida, se sumará $-\infty$, lo que resultará siempre en $-\infty$
	      sin importar hacia qué lado me mueva, representando correctamente la imposibilidad de alcanzar
	      celdas ocupadas desde esa posición.
	\item Si la celda está libre, no se suma más celdas ocupadas a $OPT$, lo cual es correcto.
	\item Si la celda está ocupada, se suma \( 1 \) al valor de $OPT$, representando una celda más que es posible alcanzar.
\end{itemize}

Por lo tanto, el resultado de \( OPT(f,c) \) es correcto en todos los casos.
\newpage

