\subsubsection{Algoritmo}
\begin{algorithm}
	\caption{CalcularOPT}
	\label{alg:calcularOPT}
	\textbf{Entrada:} Tablero $T[fila, col]$ donde cada celda $(fila, col)$ tiene el valor \{libre, ocupado, prohibido\}; $n$: tamaño del tablero \\
	\textbf{Salida:} $OPT[fila, col]$: cantidad máxima de celdas ocupadas que se pueden alcanzar
	\begin{algorithmic}[1]

		\State Crear matriz $OPT$

		\For{$k \gets 0$ \textbf{hasta} $n+1$}
		\State $OPT[k, n+1] \gets -\infty$
		\State $OPT[n+1, k] \gets -\infty$
		\State $OPT[k, 0] \gets -\infty$
		\State $OPT[0, k] \gets -\infty$
		\EndFor

		\For{$f \gets n$ \textbf{hasta} $1$} \Comment{Iterar sobre filas}
		\For{$c \gets n$ \textbf{hasta} $1$} \Comment{Iterar sobre columnas}
		\If{$(f, c) = (n, n)$} \Comment{Saltear la celda $(n, n)$}
		\State $OPT[n, n] \gets \text{celda}(n, n)$
		\EndIf
		\State $OPT[f, c] \gets \text{celda}(f, c) + \max(OPT[f+1, c], OPT[f, c+1])$
		\EndFor
		\EndFor

		\State  \Return $OPT$
	\end{algorithmic}
\end{algorithm}

Se usa el algoritmo auxiliar \texttt{celda}:

\begin{algorithm}
	\caption{celda}
	\label{alg:celda}
	\textbf{Entrada:} $(f, c)$: posición de la celda; $T[fila, col]$: tablero donde cada celda tiene un valor \{libre, ocupado, prohibido\} \\
	\textbf{Salida:} Valor de la celda en función de si está ocupada, libre o prohibida
	\begin{algorithmic}[1]
		\If{$T[f, c] = \text{ocupada}$}
		\State \Return $1$ \Comment{La celda está ocupada}
		\ElsIf{$T[f, c] = \text{libre}$}
		\State \Return $0$ \Comment{La celda está libre}
		\ElsIf{$T[f, c] = \text{prohibida}$}
		\State \Return $-\infty$ \Comment{La celda es prohibida}
		\EndIf
	\end{algorithmic}
\end{algorithm}

